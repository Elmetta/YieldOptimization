\documentclass[a4paper,12pt]{book}
%%% OPTMIZATION FOR ePRINT: onesided,openany

\usepackage[english]{babel}

\usepackage[T1]{fontenc}
\usepackage[utf8]{inputenc} %CHANGE TO latin1 IF YOU HAVE PROBLEMS WITH UMLAUTS
\usepackage{lmodern}

%% Deutsche Absätze ohne Einschub, mit Abstand
%\setlength{\parindent}{0em}
%\setlength{\parskip}{0ex plus0.5ex}

%%% ADJUST PAGE MARGINS HERE
\usepackage[
	a4paper,
	includehead,
	left=18mm,
	right=18mm,
	top=15mm,
	bottom=20mm,
	headsep=25mm,
]{geometry}

\usepackage{fancyhdr} %Package fancyhdr

% Packages für Grafiken & Abbildungen
\usepackage{graphicx}

%xcolor
\usepackage{xcolor}

\usepackage{bbm}
\usepackage{amsthm}
%\usepackage{amstext}
%\usepackage{amsfonts}
\usepackage{amssymb}
\usepackage{amsmath}
%\usepackage{latexsym}

%\usepackage[absolute,overlay]{textpos} %absolute Positionierung von boxes auf der Seite

% Paket zur Darstellung von Code
%\usepackage{listings}

% Paket hyperref sollte als letztes geladen werden
\usepackage[raiselinks=true,
  pdftex,colorlinks,bookmarks,
  bookmarks=true,
  bookmarksopenlevel=1,
  bookmarksopen=true,
  bookmarksnumbered=true,
  hyperindex=true,
  plainpages=false,
  pdfpagelabels=true,
  pdfborder={0 0 0.5}]{hyperref}

%%% Hyphenation %%%%%%%%%%%%%%%%%%%%%%%%%%%%
%\hyphenation{}
%%%%%%%%%%%%%%%%%%%%%%%%%%%%%%%%%%%%%%%%%%%%

%% -------------------------------
%% |      Globale Settings       |
%% -------------------------------
\setcounter{secnumdepth}{3} % Numbering also for \subsubsections
\setcounter{tocdepth}{3}    % Register \subsubsections in content directory

\linespread{1.5} %mehrfacher Zeilenabstand

\makeatletter
\AtBeginDocument{
%%% Setup for hyperref
  \hypersetup{
    pdftitle = {\@title},
    pdfauthor = {\@author}
  }
	
%%% Setup for fancyhdr
\pagestyle{fancy} %eigener Seitenstil
\fancyhf{} %alle Kopf- und Fußzeilenfelder bereinigen
\fancyhead[L]{\@title} %Kopfzeile links
\fancyhead[R]{\nouppercase{\leftmark}} %zentrierte Kopfzeile
\renewcommand{\headrulewidth}{0.4pt} %obere Trennlinie
\fancyfoot[L]{} 
\fancyfoot[C]{\thepage} %Seitennummer
\renewcommand{\footrulewidth}{0.4pt} %untere Trennlinie

%%Unterschied zwischen geraden/ungeraden Seiten:
%\fancyhead[OR]{} % "O" steht für "odd", also ungerade Seiten
%\fancyhead[ER]{} % "E" für "even", also gerade Seiten.
}
\makeatother


%% --- End of global Settings ---

%%%%%%%%%%%%%%%%%%%%%%%%%%%%%%%%%%%%%%%%%%%%%%%%%%%%%%%%%%%%%
%% DOCUMENT - Skript
%%%%%%%%%%%%%%%%%%%%%%%%%%%%%%%%%%%%%%%%%%%%%%%%%%%%%%%%%%%%%

\title{}
\author{}
\date{}

\begin{document}

%%%%%%%%%%%%%%%%%%%%%%%%%%%%%%%%%%%%
% Frontmatter %%%%%%%%%%%%%%%%%%%%%%
%%%%%%%%%%%%%%%%%%%%%%%%%%%%%%%%%%%%
\frontmatter

%\include{titlepage}
\maketitle

\tableofcontents

%%%%%%%%%%%%%%%%%%%%%%%%%%%%%%%%%%%%
% Mainmatter %%%%%%%%%%%%%%%%%%%%%%%
%%%%%%%%%%%%%%%%%%%%%%%%%%%%%%%%%%%%
\mainmatter

\chapter{Introduction}
% !TEX root = Vorlage_EN.tex
\section{Background}
Increasing adoption in cryptocurrencies is a phenomenon that has gained great attention over the last couple of years. Since at least 2017 Bitcoin and blockchain technology became a hyped topic and focus of many industries and investors. With a cashflow of hundreds of billions of dollars (Firmen fangen an in Bitcoin un Co. zu investieren, sowie auch in eigene Tokens). Considering the blockchain technology and cryptocurrencies in general as a megatrend, several industries are influenced and forced to respond to this new technical challenge. Companies which are implementing blockchain related applications or exchanges gaining competitive advantage in the business environment. However, they became buzzwords which use is taken out of context frequently. (drin lassen?). (Vielleicht hier diese absurden beispiele wie mit dem Fahrradladen bringen, damit man sieht wie gehyped und komplex/ unverständlich das Thema anfangs für viele war).
(Jetzt eher so auf die Regierungen eingehen: viele Restriktionen, aber auch Offenheit, irgendwas über die federal reserved irgendwas :D und wie sich das ganze auswirken könnte. Dass das ganze schwierig ist ). 
It gets down to an increasing demand in cryptocurrencies. Making a fast buck becomes a simple procedure little knowledge about finance and blockchain presupposed. Today’s society is becoming more and more attentive to this topic through news, talk-shows, social media et cetera.
Although blockchain technology is already known since the early 1990’s. With the Bitcoin Whitepaper released in 2009 it became popular and a revolution for the financial system (vielleicht noch hier warum revolution: weil probleme der dezentralität gelöst wurden etc.). Blockchains and cryptocurrencies are an opportunity for a lot of innovation in this and many other fields. An easy accessibility for both users and creators, a trustless network, privacy, new approaches of access management, possibilities of participation in governance processes of some protocols and the possibility of the reversal of inflation to name a few.
Nowadays it is not necessary to install one’s wallet through the command line anymore. Applications like Metamask makes it easy to administer one’s crypto currencies in a pretty simple way as well as financial products like … (einfachstes product zur veranschaulichung was gut nutzbar ist [Pancake oder BIFI wollte ich hier nicht erwähen, weil die ja noch später kommen c: ]). These products are based on so-called smart contracts which are holding their program code and which are stored directly on the blockchain itself so that its integrity is assured/save/immutable (vielleicht rausstreichen). 


% !TEX root = Vorlage_EN.tex
\section{Motivation}
With the Ethereum network launched in 2015 decentralized finance started its triumphal procession in 2017 with the first few implemented use cases for Ethereum. Since 2020 these use cases concentrated on more useful and actual reasonable projects (Beispiel pls, vielleicht eins der frühen 2021er). However, these projects are predestinated for more improvement and have even more potential for appending applications. Optimizer play a major role in the derivative of such and is main object of this thesis. 
(Vielleicht eigenes Yieldfarming thematisieren und als Eigenmotivation angeben, vllt auch als Anlass für den Optimizer) We would like to create an optimizer for the pancakeswap finance platform that auto-compunded and auto-staked the existing syrup pools on the platform. We want to archive an improvement for the user to save time and money. Striving passive income is a goal for many.





\chapter{Theoretical Background}
% !TEX root = Vorlage_EN.tex
\section{Blockchain}
In general

The blockchain technology is an instrument for the reproducible assurance of transactions and block creation which protects its integrity by operating in a decentralized way. The idea is to replicate the blockchain multiple times but identically by participants who can permanently compare their versions of the blockchain to each other. Therefor fraudsters who want to manipulate the blockchain for their own goods can be spotted quickly and a consensus on the legitimate transaction ledger be built. 
The growth of the blockchain is given by producing new blocks that matches the cryptographic requirements of the chain. Because all the blocks including the hash value and its signature of the previous block, previous blocks can’t be changed afterwards without changing their hash values as well. In conclusion it forms a chain of blocks which are interdependent. Next to the Block Header which includes the metadata for such block there is also the Block Data which holds the information specific for each blockchain type.
Since different blockchains differs in their usage and mechanism of producing new blocks, block size and contained information in the Block Data can diversify.
Well known mechanisms for block production are “Proof of Work” and “Proof of Stake”.

Classification

Blockchains can be distinguished in a few different types which differ in anonymity/ pseudonymity and level of authorization they provide. The first three types are concerned with the in anonymity/ pseudonymity while the last two apply to the authorization level.
Public: Public Blockchains managing themselves in a decentralized, independent way like in the explanation above where multiple independent individuals administer the blockchain. The participants don’t know each other and have a certain amount of mutual distrust.
Private: In case of private blockchains the blockchain is managed by one organisation. Read permissions may be public but writing to the blockchain is a reserve privilege to internal organisation’s members \cite{Shermin2017}.
Federated: Rather unrecognized are federated blockchains which are organized by groups with a pre-defined set of nodes. According to Voshmgir and Kalinov it can be described by “a consortium of 15 financial institutions, each of which operates a node and of which 10 must sign every block in order for the block to be valid” \cite{Shermin2017}.
Permissionless: Permissionless prescribe no further conditions for participation. Everyone who has the resources for validating a new block can take part without proving one’s identity. 
Permissioned: In contrast permissioned blockchain tie their members to certain prerequisites which must be satisfied first.
Several combinations of these types are possible. For example, a private permissioned blockchain needs its validators to be an employee or a selected agent by the company. On a public permissionless blockchains like Bitcoin everyone can take part in producing new blocks without restrictions. In case of a public permissioned one like  all “Proof of Stake” chains everyone can validate transactions if they fulfil particular criteria~\cite{Shermin2017}. For the new Ethereum 2.0 version (which is not released up to this date Dez 2021) one has to have at least 32 ETH to be a validator.
% !TEX root = Vorlage_EN.tex
\section{Smart Contracts}
Grech 2017 describes smart contracts as “effectively small computer programmes stored on a blockchain, which will perform a transaction under specified conditions”~\cite{GrechAlexander2017}. Smart contracts are triggered automatically if this condition applies. According to Mitschele~\cite{Mitschele2019} the concept is comparable to a vending machine, which spends the drink after insert a certain amount of money or gives your change back if it was not enough or it ran out of the product. Even though the contract is self-executing usually there is an instance of activation. For example, a spender who wants to interact with the smart contract must provide the needed amount of money for gas and the purchase itself so the contract can check for this condition and executes itself as a consequence. Smart contracts can also be triggered by other contracts.
Because of the “conditions, and Ether balance” which are administered by the contracts themselves Meinel and Gayvoronskaya pointed out that smart contracts can rather be described as “autonomous agents” instead of “cryptographic “boxes” with specific values that can only be unlocked if certain conditions are met.“~\cite{meinel2018blockchain}
No matter what definition is used, smart contracts comply the characteristics of self-verifying, self-executing and tamper resistant~\cite{Shermin2017}. Furthermore smart contracts can "Turn legal obligations into automated processes", "Guarantee greater degree of security", "Decreasing reliance on trusted intermediaries" and "Lower transaction costs"~\cite{Shermin2017}.

The use case of smart contracts are diverse/manifold?. It range from simple governmental registries up to provisioning of external informations to execute the contract thorugh oracles. Performing digital value exchange or NFT minting is one of the simplest implementations of smart contracts and the complexity can rise according to its use case.
For this Thesis the implementation of smart contracts as decentralized Applications (dApps) is important. According to Vitalis Buterin, decentralized apps are the combination of “low-level business logic components”~\cite{buterin2014next} and “high-level graphical user“~\cite{buterin2014next} interfaces. Which brings together the smart contract backend and the user interface frontend.


Since the contract is stored on the blockchain one can not change the code if it is once deployed. Therefor it is highly important to test the code onto every possible occurrence of errors and eliminate them before deploying. 
“For smart contracts to become a viable commercial tool, they must be not only technically but also legally enforceable“~\cite{mik2017smart}.



% !TEX root = Vorlage_EN.tex
\section{Ethereum Chain}
When Bitcoin established itself for a couple of years it became clear that there is a lot of potential in optimizing the concept of the Bitcoin technology. One projects which tried to archive that in terms of efficienty, scalability and speed was Ethereum which idea arises in 2013. After a developing span of nearly two years Ethereum was launched in 2015 and overcomes several problems of the cryptocurrencies known back then. The goal of the Ethereum network was to build a multitool working like a general operating system which can run arbitrary applications written by whoever can program smart contracts~\cite{Vitalik2015}.
In contrast to other known blockchains, Ehtereum implements two types of accounts, user accounts and contract accounts~\cite{buterin2014next}. This made Ethereum the first open ledger peer-to-peer network for deploying and interacting with smart contract code. With the build-in programming language known as Solidity scripting language limitations of Bitcoin were handled and a Turing-completeness was implemented in the Ethereum Virtual Machine (EVM)~\cite{Vujicic2018, buterin2014next}.

The EVM 



Binance 

One of the rather disadvantages of the Ethereum Blockchain not mentioned in the last chapter is that its block producting mechanismn is based on "Proof of Work". Even though the developers are working on the newer Ehtereum 2.0 version currently to implement "Proof-of-Stake" it could be witnessed in the last Crypto Marktet Cycel of 2021 that a lot of people were reliant on lower gas fees.

% !TEX root = Vorlage_EN.tex
\section{DEFI}
Definition, was sit es?
Wie funktioniert es?
Anwendungsfälle? (heute und zukünftig)
Welche Probleme Löst es
staking, landing, shorten, longen
dezentrale crypto-börsen (wobei das vielleicht to much ist)
wie groß ist der markt? (bitte mit zahlen belegen)
Risiken


\subsection{Syrup Pools}
Was ist pancake swap?
für was sind die syrup pools
...läuft auf BSC Chain...
warum sind diese sinnvoll/ Mehrwert?
wie funktionierts
APR erklärt und warum so hoch und dann so tief
Woher kommen die reward tokens


\subsection{Liquidity Pools}

\subsection{Yield Farming}
\subsubsection{Voraussetzungen Yield Aggregation}
\subsubsection{Topographie (Yield Aggregator „allgemein“)}
\subsubsection{Beispiel: yearn.finance}



%%%MATHE%%%
% https://ei.hs-duesseldorf.de/personen/braun/lehre/Documents/LaTeX%20SS17/Latex%2008%20-%20Formeln.pdf

\begin{equation}
n_1 \sin \alpha = n_2 \sin \beta
S_2>e_1\cdot e_2
\end{equation}
\begin{equation}
E_1
\end{equation}
\begin{equation}
E1
\end{equation}
\begin{equation}
<>+-/()!= \cdot \alpha \beta \gamma \sin \log
\end{equation}
\begin{equation}
R^{M}
\end{equation}
\begin{equation}
R^{M \cdot R^L}
\end{equation}



%%%MATHE%ENDE%%%


%%%%%%%%%%%%%%%%%%%%%%%%%%%%%%%%%%%%
% Backmatter %%%%%%%%%%%%%%%%%%%%%%%
%%%%%%%%%%%%%%%%%%%%%%%%%%%%%%%%%%%%
\backmatter
\appendix

\bibliographystyle{IEEEtran}
\bibliography{./references/references}

\end{document}
